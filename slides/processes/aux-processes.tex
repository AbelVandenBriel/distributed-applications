\section{Elixir Processes}

\frame{\tableofcontents[currentsection]}

\begin{frame}
    \frametitle{Warning}
    \begin{itemize}
        \item Elixir abstracts away OS concepts such as processes and threads
        \item It can still be useful to know how exactly Elixir concepts map on OS concepts
        \item {\bfseries\color{red} Elixir reuses same terms but assigns different meaning!}
    \end{itemize}
\end{frame}

\begin{frame}
    \frametitle{Elixir Processes}
    \code[language=elixir]{hello-world.exs}
    \begin{itemize}
        \item In other language
              \begin{itemize}
                \item A \emph{thread} executes this line of code
                \item After this, the thread ends
              \end{itemize}
        \item In Elixir
              \begin{itemize}
                \item A \emph{process} executes this line of code
                \item After this, the process ends
              \end{itemize}
        \item Only difference: terminology
    \end{itemize}
\end{frame}

\begin{frame}
    \frametitle{Thread Creation in \csharp}
    \code[language=csharp,font=\small,width=.95\linewidth]{create-thread.cs}
    \begin{itemize}
        \item New thread needs entry point
        \item Constructor is given function
        \item Thread will call \texttt{ThreadProc}
        \item Thread dies when \texttt{ThreadProc} is done
    \end{itemize}
\end{frame}

\begin{frame}
    \frametitle{Process Creation in Elixir}
    \code[language=elixir,font=\small,width=.95\linewidth]{spawn-process.exs}
    \begin{itemize}
        \item New thread needs entry point
        \item \texttt{spawn} is given function
        \item Process will call \texttt{threadfunc}
        \item Process dies when \texttt{threadfunc} is done
    \end{itemize}
\end{frame}

\begin{frame}
    \frametitle{So What's The Big Deal?}
    \begin{itemize}
        \item Difference between Elixir and \csharp?
        \item Currently limited to aesthetics
        \item What's up with that, Elixir? Why do you even exist?
    \end{itemize}
\end{frame}

\begin{frame}
    \frametitle{Communication Between Threads in \csharp}
    \code[language=csharp,font=\small,width=.95\linewidth]{communication.cs}
\end{frame}

\begin{frame}
    \frametitle{Communication Between Processes in Elixir}
    \begin{center}
        \Huge ...
    \end{center}
\end{frame}

\begin{frame}
    \frametitle{Communication Between Processes in Elixir}
    \begin{itemize}
        \item Elixir is purely functional
        \item There is no state
        \item A newborn process can receive data, but it never changes
        \item Communication is in essence change
        \item There is no way to communicate without state
        \item Are we in trouble?
    \end{itemize}
\end{frame}

\begin{frame}
    \frametitle{Communication Between Processes in Elixir}
    \begin{itemize}
        \item Elixir introduces a tiny bit of state
        \item It's fully handled internally by Elixir
        \item No need for synchronization due to shared state
    \end{itemize}
\end{frame}

\begin{frame}
    \frametitle{Message Passing}
    \begin{center}
        \begin{tikzpicture}[process/.style={drop shadow,fill=red!50},
                            process header/.style={black,opacity=0.5},
                            mailbox/.style={drop shadow,fill=blue!50}]
            \begin{scope}
                \coordinate (process a center) at (2,2.5);
                \coordinate (process a queue) at (0.25,-0.25);
                \draw[process] (0,0) rectangle ++(4,5);
                \node[process header,minimum width=4cm,anchor=north west] at (0,5) { Process A };
                \draw[mailbox] (0.5,-0.5) rectangle ++(3,1);
            \end{scope}

            \begin{scope}[xshift=4.5cm]
                \coordinate (process b center) at (2,2.5);
                \coordinate (process b queue) at (0.75,-0.4);
                \draw[process] (0,0) rectangle ++(4,5);
                \node[process header,minimum width=4cm,anchor=north west] at (0,5) { Process B };
                \draw[mailbox] (0.5,-0.5) rectangle ++(3,1);
            \end{scope}

            \only<2>{
                \node[font=\tt] at (process a center) {
                    send(B, :hello)
                };
            }

            \only<3>{
                \node[font=\tt] (send) at (process a center) {
                    send(B, :hello)
                };

                \envelope[position=process b queue]

                \draw[-latex,ultra thick] (send) -- ($ (process b queue) + (0,0.4) $);
            }

            \only<4>{
                \node[font=\tt] (receive) at (process b center) {
                    receive
                };

                \envelope[position=process b queue]
            }

            \only<5>{
                \node[font=\tt] (receive) at (process b center) {
                    receive
                };

                \envelope[position=process b queue]
                \draw[-latex,ultra thick] ($ (process b queue) + (0.5,0.4) $) -- (receive);
            }
        \end{tikzpicture}
    \end{center}
\end{frame}

\begin{frame}
    \frametitle{Message Passing}
    \code[language=elixir,font=\small]{message-passing.exs}
    \begin{tikzpicture}[overlay,remember picture]
        \only<2>{
            \codeoverlinex{spawn}
            \node[note,anchor=south west] at ($ (spawn1) + (0,1) $) {Creates a new process};
        }
        \only<3>{
            \codeunderlinex{child}
            \node[note2,anchor=north west] (note) at ($ (child1) + (0,-0.75) $) {New process executes this function};
            \draw[note arrow] (note) -- (child);
            \codeoverlinex{drive}
            \node[note,anchor=south west] (note) at ($ (drive1) + (0,0.75) $) {Meanwhile, parent process drives.};
            \draw[note arrow] (note) -- (drive);
        }
        \only<4>{
            \codeunderlinex{annoy}
            \node[note2,anchor=north west] (note) at ($ (annoy1) + (0,-0.75) $) {Child process prints message};
            \draw[note arrow] (note) -- (annoy);
            \codeunderlinex{drive}
            \node[note,anchor=north west] (note) at ($ (drive1) + (0,-0.75) $) {Goes vroom!};
            \draw[note arrow] (note) -- (drive);
        }
        \only<5>{
            \codeunderlinex{receive}
            \node[note2,anchor=north west] (note) at ($ (receive) + (0,-0.75) $) {Child process checks mailbox};
            \draw[note arrow] (note) -- (receive);
            \codeunderlinex{drive}
            \node[note,anchor=north west] (note) at ($ (drive1) + (0,-0.75) $) {Vrooooooom!};
            \draw[note arrow] (note) -- (drive);
        }
        \only<6>{
            \codeoverlinex{timeout}
            \node[note2,anchor=south west] (note) at ($ (timeout) + (0,0.75) $) {
                \parbox{8cm}{
                    Normally, \texttt{receive} waits for a message to arrive indefinitely. In this case,
                    we are dealing with an impatient process. It waits at most 1ms for a message to arrive.
                }
            };
            \draw[note arrow] (note) -- (timeout);
            \codeunderlinex{drive}
            \node[note,anchor=north west] (note) at ($ (drive1) + (0,-0.75) $) {EEeeeeeeeeeeee!};
            \draw[note arrow] (note) -- (drive);
        }
        \only<7>{
            \codeoverlinex{rec}
            \node[note2,anchor=south] (note) at ($ (rec) + (0,0.75) $) {
                \parbox{5cm}{
                    No \texttt{:arrive} message yet. Function calls itself.
                }
            };
            \draw[note arrow] (note) -- (rec);
            \codeunderlinex{drive}
            \node[note,anchor=north west] (note) at ($ (drive1) + (0,-0.75) $) {(Forgot Grampa)};
            \draw[note arrow] (note) -- (drive);
        }
        \only<8>{
            \codeunderlinex{annoy}
            \node[note2,anchor=north] (note) at ($ (annoy) + (0,-0.75) $) {
                Back to square one.
            };
            \draw[note arrow] (note) -- (annoy);
            \codeunderlinex{drive}
            \node[note,anchor=north west] (note) at ($ (drive1) + (0,-0.75) $) {Moorv moorv moorv!};
            \draw[note arrow] (note) -- (drive);
        }
        \only<9>{
            \codeunderlinex{child}
            \codeoverlinex{rec}
            \node[note2,anchor=north west] (note) at ($ (child) + (0,-0.5) $) {
                Loop goes on and on.
            };
            \draw[note arrow] (note) -- (child);
            \draw[note arrow] (note) -- (rec);
            \codeunderlinex{drive}
            \node[note,anchor=north west] (note) at ($ (drive1) + (0,-0.75) $) {Arrives back home.};
            \draw[note arrow] (note) -- (drive);
        }
        \only<10>{
            \codeoverlinex{send}
            \node[note,anchor=south west] (note) at ($ (send1) + (0,0.75) $) {Tells child process we arrived.};
            \draw[note arrow] (note) -- (send);
        }
        \only<11>{
            \codeunderlinex{receive}
            \node[note2,anchor=north west] (note) at ($ (receive) + (0,-0.75) $) {Child process checks mailbox once more};
            \draw[note arrow] (note) -- (receive);
        }
        \only<12>{
            \codeunderlinex{msg}
            \node[note2,anchor=north west] (note) at ($ (msg) + (0,-0.75) $) {It finds an \texttt{:arrive} message!};
            \draw[note arrow] (note) -- (msg);
        }
        \only<13>{
            \codeunderlinex{finally}
            \node[note2,anchor=north] (note) at ($ (finally) + (0,-0.75) $) {Emits final cry. Child process ends. Silence at last.};
            \draw[note arrow] (note) -- (finally);
        }
    \end{tikzpicture}
\end{frame}
